% Created 2016-10-05 Wed 20:05
\documentclass[12pt,letterpaper]{article}
\usepackage[utf8]{inputenc}
\usepackage[T1]{fontenc}
\usepackage{fixltx2e}
\usepackage{graphicx}
\usepackage{longtable}
\usepackage{float}
\usepackage{wrapfig}
\usepackage{rotating}
\usepackage[normalem]{ulem}
\usepackage{amsmath}
\usepackage{textcomp}
\usepackage{marvosym}
\usepackage{wasysym}
\usepackage{amssymb}
\usepackage{hyperref}
\tolerance=1000
\usepackage[letterpaper]{geometry}
\author{Taylor Foxhall}
\date{\today}
\title{Intro to Security Homework 2}
\hypersetup{
  pdfkeywords={},
  pdfsubject={},
  pdfcreator={Emacs 25.1.1 (Org mode 8.2.10)}}
\begin{document}

\maketitle
\section{Problem 1}
\label{sec-1}
Alice's RSA public key is (N, e) = (33, 3), her private key is d
= 7.
\subsection{Part A}
\label{sec-1-1}
What is the ciphertext C if Bob encrypts M = 19 using Alice's
public key?

$$ C = M^e \mod N $$
$$ M = C^d \mod N $$

So $C = 19^3 \mod 33 = 28$. Alice can decrypt this only if she
knows N and d. Fortunately, N is known to everyone and d is known
only to Alice. So $M = 28^7 \mod 33 = 19$, as we expect it to be.
\subsection{Part B}
\label{sec-1-2}
If Alice signs S, what is S? And how can Bob verify that Alice
signed S?

Alice can sign M by using her private key to encrypt M. So, $S =
   25^7 \mod 33 = 31$, and then when she send M and S to Bob, he can
try to decrypt S using her public key. Then he will check if D(S)
and M match. We can compute $D(S) = 31^3 \mod 33 = 25 = M$. Since
Alice is the only one who knows her private key, S must have been
signed by Alice.
\section{Problem 2}
\label{sec-2}
Alice wants to send Bob $g^a \mod p$ to Bob as in the Diffie-Hellman
protocol. Bob wants the secret to be X. Can Bob choose his
Diffie-Hellman value, $g^b \mod p$, such that the shared secret is
X?

In order for this to be possible $X = g^{ab} \mod p$. So Bob must
select b such that $X = (g^a)^b \mod p$. By doing a little modular
arithmetic we can calculate b as
\begin{align*}
logX &= blog(g^a) \mod p \\
b &= logX * (log(g^a))^{-1} \mod p
\end{align*}
So provided Bob can solve $log(g^a)^{-1} \mod p$ he can choose b
according to this restriction.
\section{Problem 3}
\label{sec-3}
\subsection{Part A}
\label{sec-3-1}
Given the curve $E: y^2 = x^3 + 7x + b$ find $b$ so that $P = (4,
   5)$ lies on it.
\begin{align*}
5^2 &= 4^3 + 7(4) + b \mod 11 \\
25 &= 92 + b \mod 11 \\
3 &= 4 + b \mod 11 \\
b &= 10 \mod 11
\end{align*}
\subsection{Part B}
\label{sec-3-2}
List all points on $E$.

Since E is in the integers mod 11, we can check all integers $x$, $0
   \leq x < 11$ for solutions.
\begin{align*}
y^2 &= (0)^3 + 7(0) + 10 = 10 \mod 11 & \text{No solution.} \\
y^2 &= (1)^3 + 7(1) + 10 = 7 \mod 11 & \text{No solution.} \\
y^2 &= (2)^3 + 7(2) + 10 = 10 \mod 11 & \text{No solution.} \\
y^2 &= (3)^3 + 7(3) + 10 = 3 \mod 11 & y=5,6 \\
y^2 &= (4)^3 + 7(4) + 10 = 3 \mod 11 & y=5,6 \\
y^2 &= (5)^3 + 7(5) + 10 = 5 \mod 11 & y=4,7 \\
y^2 &= (6)^3 + 7(6) + 10 = 4 \mod 11 & y=2,9 \\
y^2 &= (7)^3 + 7(7) + 10 = 6 \mod 11 & \text{ No solution.} \\
y^2 &= (8)^3 + 7(8) + 10 = 6 \mod 11 & \text{ No solution.} \\
y^2 &= (9)^3 + 7(9) + 10 = 10 \mod 11 & \text{ No solution.} \\
y^2 &= (10)^3 + 7(10) + 10 = 2 \mod 11 & \text{ No solution.}
\end{align*}
So the points are $\{
   (3,5),(3,6),(4,5),(4,6),(5,4),(5,7),(6,2),(6,9), \textbf{O} \}$.
\subsection{Part C}
\label{sec-3-3}
Find the sum of $(4,5)+(5,4)$ on $E$.

\begin{align*}
c &= (4 - 5) * (5 - 4)^{-1} \mod 11 \\
&= -1 * (1)^{-1} \mod 11 \\
&= 10 * 1 \mod 11 \\
&= 10 \mod 11 \\
\\
x_3 &= 10^2 - 4 - 5 \mod 11 \\
&= 91 \mod 11 \\
&= 3 \mod 11 \\
\\
y_3 &= 10(4 - 3) - 5 \mod 11 \\
&= 5 \mod 11
\\
(4,5) + (5,4) &= (x_3, y_3) = (3, 5) \\
\end{align*}
\subsection{Part D}
\label{sec-3-4}
Find $3(4,5)$.

Since there's no direct way to compute $3(4,5)$ we must compute
$(4,5) + (4,5) + (4,5)$ iteratively. We start with $(4, 5) + (4,5)$. 
\begin{align*}
c &= (3(4)^2 + 7) * (2(5))^{-1} \mod 11 \\
&= 0 \mod 11 \\
\\
(4,5) + (4,5) &= (0 - 4 - 4, 0(4 - x_3) - 5) \mod 11 \\
&= (3, 6) \mod 11
\end{align*}
And then we can calculate $(3, 6) + (4,5)$.
\begin{align*}
c &= (5-6) * (4-3)^{-1} \mod 11 \\
&= 10 \mod 11 \\
\\
(3,6) + (4,5) &= (10^2 - 3 - 4, 10(3 - x_3) - 6) \\
&= (5, 10(9) - 6) \\
&= (5, 7)
\end{align*}
So $3(4,5) = (5,7)$.
\section{Problem 4}
\label{sec-4}
Alice could find another key F, $F \neq K$, such that the output of
decrypting Y with F would output an X different than what Bob told
her if Bob's guess was correct. Bob has no way of knowing that she
changed the keys, and so Alice could win everytime provided she
could find a key that changes what X is. The way to fix that is for
Alice to also send h(R) over along with Y, and when Bob decrypts Y
with Alice's key he can verify that the hash Alice sent him matches
the hash of the R he just decrypted using her key. It would be
significantly more difficult for Alice to send him a different key
that preserves h(R) but changes X in the decryption.
% Emacs 25.1.1 (Org mode 8.2.10)
\end{document}
