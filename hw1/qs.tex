% Created 2016-09-18 Sun 23:53
\documentclass[11pt]{article}
\usepackage[utf8]{inputenc}
\usepackage[T1]{fontenc}
\usepackage{fixltx2e}
\usepackage{graphicx}
\usepackage{longtable}
\usepackage{float}
\usepackage{wrapfig}
\usepackage{rotating}
\usepackage[normalem]{ulem}
\usepackage{amsmath}
\usepackage{textcomp}
\usepackage{marvosym}
\usepackage{wasysym}
\usepackage{amssymb}
\usepackage{hyperref}
\tolerance=1000
\author{Taylor Foxhall}
\date{\today}
\title{Intro to Security HW 1}
\hypersetup{
  pdfkeywords={},
  pdfsubject={},
  pdfcreator={Emacs 25.1.1 (Org mode 8.2.10)}}
\begin{document}

\maketitle
\section{Problem 1}
\label{sec-1}
An affine cipher is a type of simple substitution where each letter
is encrypted according to the following rule $c = (a p + b)
  \mod 26$. Here, p, c, a, and b are each numbers in the range of 0 to
25, where p represents the plaintext letter, c the ciphertext
letter, and a and b are constants. For the plaintext and ciphertext,
0 corresponds to "a," 1 corresponds to "b," and so on. Consider the
ciphertext QJKES REOGH GXXRE OXEO, which was generated using an
affine cipher. Determine the constants a and b and decipher the
message. Hint: Plaintext "t" encrypts to ciphertext "H" and
plaintext "o" encrypts to ciphertext “E.”
\subsection{Answer}
\label{sec-1-1}
Given that "t" encrypts to "H", and "o" encrypts to "E".
\begin{align*}
7 &= 19a + b \mod 26 \\
4 &= 14a + b \mod 26
\end{align*}
Subtracting them gives us
\begin{align*}
3 &= 5a \mod 26 \\
5^{-1} &= 21 \mod 26 \\
a &= 21*3 \mod 26 \\
a &= 11 \\
\\
7 &= (11)(19) + b \mod 26 \\
7 &= 209 + b \mod 26 \\
b &= -202 \mod 26 \\
b &= 6
\end{align*}
So with a little python script we can decipher the text.
\begin{verbatim}
def encrypt_letter(p, a, b):
    pp = ord(p) - ord('a')
    c = (pp*a + b) % 26
    return chr(c + ord('A'))

A = 11
B = 6
CIPHERTEXT = "QJKESREOGHGXXREOXEO"
ALPHABET = "abcdefghijklmnopqrstuvwxyz"

# Create a mapping from ciphertext letters to plaintext letters
decrypt_letter = {encrypt_letter(p, A, B): p for p in ALPHABET}

# Lookup every ciphertext's letter and join the whole result
return "".join(decrypt_letter[c] for c in CIPHERTEXT)
\end{verbatim}

Which yields us $P = ifyoubowatallbowlow$, or with appropriate
spacing and punctuation: "If you bow at all, bow low."
\section{Problem 2}
\label{sec-2}
Consider a Feistel cipher with four rounds. Then the plaintext is
denoted as P = (L0, R0) and the corresponding ciphertext is C = (L4,
R4). What is the ciphertext C, in terms of L0, R0, and the subkey, for
each of the following round functions? (You should get the most
concise solution.)
\begin{itemize}
\item F(Ri-1, Ki) = 0
\item F(Ri-1, Ki) = Ri-1
\item F(Ri-1, Ki) = Ki
\item F(Ri-1, Ki) = Ri-1 ⊕ Ki
\end{itemize}
(Note        that for each of cases A – D, the cipher uses four rounds.)
\subsection{Answer}
\label{sec-2-1}
\subsubsection{F(R$_{\text{i-1}}$, K$_{\text{i}}$) = 0}
\label{sec-2-1-1}
\begin{align*}
L_1 &= R_0 \\
R_1 &= R_0 \oplus 0 = L_2 \\
R_2 &= R1 \oplus 0 = L_3 = R_0 \oplus 0 \\
R_3 &= R2 \oplus 0 = L_4 = R_0 \oplus 0 \\
R_4 &= R_3 \oplus 0 = R_0 \oplus 0
\end{align*}
\subsubsection{F(R$_{\text{i-1}}$, K$_{\text{i}}$) = R$_{\text{i-1}}$}
\label{sec-2-1-2}
\begin{align*}
L_1 &= R_0 \\
R_1 &= R_0 \oplus R_0 = L_2 = 0 \\
R_2 &= R_1 \oplus R_1 = L_3 = 0 \\
R_3 &= R_2 \oplus R_2 = L_4 = 0\\
R_4 &= R_3 \oplus R_3 = 0
\end{align*}
\subsubsection{F(R$_{\text{i-1}}$, K$_{\text{i}}$) = K$_{\text{i}}$}
\label{sec-2-1-3}
\begin{align*}
L_1 &= R_0 \\
R_1 &= R_0 \oplus K_1 = L_2 \\
R_2 &= R_1 \oplus K_2 = L_3 = R_0 \oplus K_1 \oplus K_2 \\
R_3 &= R_2 \oplus K_3 = L_4 = R_0 \oplus K_1 \oplus K_2 \oplus K_3 \\
R_4 &= R_3 \oplus K_4 = R_0 \oplus K_1 \oplus K_2 \oplus K_3 \oplus K_4
\end{align*}
\subsubsection{F(R$_{\text{i-1}}$, K$_{\text{i}}$) = R$_{\text{i-1}}$ ⊕ K$_{\text{i}}$}
\label{sec-2-1-4}
\begin{align*}
L_1 &= R_0 \\
R_1 &= R_0 \oplus R_0 \oplus K_1 = L_2 \\
R_2 &= R_1 \oplus R_1 \oplus K_2 = L_3 = K_2 \\
R_3 &= R_2 \oplus R_2 \oplus K_3 = L_4 = K_3 \\
R_4 &= R_3 \oplus R_3 \oplus K_4 = K_4
\end{align*}
\section{Problem 3}
\label{sec-3}
Suppose that we use a block cipher to encrypt according to the
rule:
\begin{align*}
C0&=IV⊕E(P0, K), \\
C1&=C0⊕E(P1, K), \\
C2&=C1⊕E(P2, K), \\
...
\end{align*}
\subsection{Answer}
\label{sec-3-1}
\subsubsection{What is the corresponding decryption rule?}
\label{sec-3-1-1}
\begin{align*}
P_0 &= D(IV \oplus C_0, K) \\
P_i &= D(C_{i-1} \oplus C_i, K) & i > 0
\end{align*}
\subsubsection{Give two security disadvantages of this mode as compared to CBC mode.}
\label{sec-3-1-2}
Say $P = P_0P_1P_2P_3$ and in particular $P_1=P2$. Then,

\[ C_0 = IV \oplus E(P_0, K) \]
\[ C_2 = IV \oplus E(P_0, K) \oplus E(P_1, K) \oplus E(P_2, K) = IV \oplus E(P_0, K) \]

This is a compromise of confidentiality, because the same
plaintext blocks can yield the same ciphertext blocks, similar to
the issues with ECB. At the same time, say Trudy switches $C_0$
and $C_2$. Then Bob, who receives the message, will get $P_{trudy}
    = P_1P_2P_1P_3$. This is a compromise of integrity, as the message
can be rearranged and altered given certain matching plaintexts.
% Emacs 25.1.1 (Org mode 8.2.10)
\end{document}
